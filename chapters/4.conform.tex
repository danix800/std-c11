\chptr{合规性}{conform}

\paragraph{}
在本国际标准中,``应该''解释为实现或程序的要求;反之,``不应''解释为禁止。

\paragraph{}
如果违反了出现在约束或运行时约束之外的``应该''或``不应''要求,则行为未定义。未定
义行为在本国际标准中以``未定义行为''或省略任何明确的行为定义来表示。这三者在强调
上没有区别;它们都描述了``未定义的行为''。

\paragraph{}
在所有其他方面正确的程序,在正确的数据上操作,包含未指明的行为,应为正确的程序,
并按照\ref{env.concept.exec.exec}的要求执行。

\paragraph{}
该实现不应成功地翻译包含\texttt{\#error}预处理指令的预处理翻译单元,除非它是被条
件包含跳过的组的一部分。

\paragraph{}
\textit{严格合规程序}只能使用本国际标准中规定的语言和库的那些特征。\footnote{严
格合规的程序可以使用条件特性(见\ref{lang.ppdir.predef.cond}),前提是使用由使用
相关宏的适当条件包含预处理指令保护。例如:                                    \\
\mbox{\qquad\texttt{\#ifdef \_\_STDC\_IEC\_559\_\_ /* FE\_UPWARD defined */}} \\
\mbox{\qquad\qquad\texttt{/* ... */}}                                         \\
\mbox{\qquad\qquad\texttt{fesetround(FE\_UPWARD);}}                           \\
\mbox{\qquad\qquad\texttt{/* ... */}}                                         \\
\mbox{\qquad\texttt{\#endif}}
} 不应产生依赖于任何未指定、未定义或实现定义行为的输出,且不应超过任何最低实施限
制。

\paragraph{}
\textit{合规实现}的两种形式是宿主式和自由式。\textit{合规宿主实现}应接受任何严格
合规程序。\textit{合规自由式实现}应接受任何严格合规的程序,在该程序中,库章
(第7章)中指定的功能的使用仅限于标准头的内容<float.h>、<iso646.h>、<limits.h>、
<stdagn.h>、<stdarg.h>、<stdbool.h>、<stddef.h>、<stdint.h>和<stdnoreturn.h>。
合规实现可以有扩展(包括附加的库函数),前提是不会改变任何严格合规程序的行为。
\footnote{这意味着合规实现只保留本国际标准中明确保留的标识符。}

\paragraph{}
\textit{合规程序}是合规实现可接受的程序。\footnote{严格合规程序旨在在合规实现中
最大限度地可移植。合规程序可能取决于合规实现的不可移植特性。}

\paragraph{}
实施应附有文档,定义所有实现定义和特定语言环境的特征以及所有扩展。

\fwdref{6.10.1,6.10.5,7.7,7.9,7.10,7.15,7.16,7.18,7.19,7.20,7.23}
