\chptr{合规性}{conform}

\paragraph{}
本标准中,``应该''理解为对实现或程序的要求;反之,``不应''理解为禁止。

\paragraph{}
如果违反约束或运行时约束之外的``应该''或``不应''要求,则行为未定义。未定义行为在
本标准中以``未定义行为''或省略任何明确的行为定义来表示。这三者在强调上没有区别;
它们都描述了``未定义的行为''。

\paragraph{}
在所有其他方面都正确,操作于正确的数据上,包含未指明的行为的程序,应该是一个正确
的程序,并按照\ref{env.concept.exec.exec}的要求执行。

\paragraph{}
实现不应该成功地翻译包含\texttt{\#error}预处理指令的预处理翻译单元,除非它是被条
件包含跳过的一部分。

\paragraph{}
\textit{严格合规程序}应该只使用本标准中规定的语言和标准库的特性。\footnote{严格
合规程序可以使用条件特性(见\ref{lang.ppdir.predef.cond}),前提是使用相关宏的条
件包含预处理指令加以保护。例如:                                              \\
\mbox{\qquad\texttt{\#ifdef \_\_STDC\_IEC\_559\_\_ /* FE\_UPWARD defined */}} \\
\mbox{\qquad\qquad\texttt{/* ... */}}                                         \\
\mbox{\qquad\qquad\texttt{fesetround(FE\_UPWARD);}}                           \\
\mbox{\qquad\qquad\texttt{/* ... */}}                                         \\
\mbox{\qquad\texttt{\#endif}}
} 不应该产生依赖于任何未指明、未定义或实现定义行为的输出,且不应超过任何最小实现
限制。

\paragraph{}
\textit{合规实现}的两种形式是宿主式和自由式。\textit{合规宿主实现}应接受任何严格
合规程序。\textit{合规自由式实现}应接受任何严格合规的程序,在该程序中,标准库
(第\ref{lib}章)中指定特性应仅限于使用标准头<float.h>、<iso646.h>、<limits.h>、
<stdagn.h>、<stdarg.h>、<stdbool.h>、<stddef.h>、<stdint.h>和<stdnoreturn.h>中的
内容。合规实现可以有扩展(包括附加的库函数),前提是不会改变任何严格合规程序的行
为。\footnote{这意味着合规实现只保留本标准中明确保留的标识符。}

\paragraph{}
\textit{合规程序}是合规实现可接受的程序。\footnote{严格合规程序旨在合规实现之间
最大限度地可移植性。合规程序可能依赖于合规实现的不可移植特性。}

\paragraph{}
实现应附有文档,说明所有实现定义和特定语言环境的特性以及所有扩展。

\fwdref{6.10.1,6.10.5,7.7,7.9,7.10,7.15,7.16,7.18,7.19,7.20,7.23}
