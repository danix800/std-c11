\chptr{环境}{env}

\paragraph{}
一个实现在两个数据处理系统环境中翻译C源文件并执行C程序,这两个环境在本国际标准中
称为\textit{翻译环境}和\textit{执行环境},其特性定义和约束执行根据合规实现的语法
和语义规则构造合规C程序结果。

\fwdref{本章中仅列举许多可能的引用中的一些。}

\sect{概念模型}{env.concept}

\ssect{翻译环境}{env.concept.trans}

\sssect{程序结构}{evn.concept.trans.struct}
\paragraph{}
C程序不需要同时翻译。在本国际标准中,程序文本以称为\textit{源文件}(或\textit{预
处理文件})的单位保存。源文件以及通过预处理指令\texttt{\#include}包含的所有头文
件和源文件被称为\textit{预处理翻译单元}。经过预处理后,预处理翻译单元称为
\textit{翻译单元}。以前翻译过的翻译单元可以单独保存,也可以保存在库中。程序的独
立翻译单元通过(例如)调用其标识符具有外部链接的函数、操作其标识符具有外部链接的
对象或操作数据文件来进行通信。翻译单元可以单独翻译,然后再链接以生成可执行程序。

\fwdref{6.2.2,6.9,6.10。}

\sssect{翻译阶段}{env.concept.trans.phase}
\paragraph{}
翻译语法规则的优先顺序由以下阶段指定。\footnote{实现应表现的如同这些单独的阶段仍
然发生,即使在实践中许多阶段通常是折叠在一起的。源文件、翻译单元和翻译后的单元不
必存储为文件,也不需要这些实体和任何外部表示之间存在任何一一对应的关系。描述只是
概念性的,没有指定任何特定的实现。}
\begin{enumerate}
  \item{如必要,物理源文件多字节字符将以实现定义的方式映射到源字符集(为行结束符
    引入新行字符)。三字符序列被相应的单字符内部表示替换。}
  \item{删除反斜杠字符(\tm{\bs})紧接新行字符的每个实例,将物理源行拼接成逻辑源
    行。只有物理源行上的最后一个反斜杠才能成为此拼接的一部分。非空的源文件应以新
    行字符结尾,在进行任何此类拼接之前,该新行字符不得紧跟在反斜杠字符后面。}
  \item{源文件被分解为预处理标记\footnote{如\ref{lang.lex}所述,将源文件的字符划
    分为预处理标记的过程取决于上下文。例如,请参见在\tm{\#include}预处理指令中对
    \tm{<}的处理。} 和空白字符序列(包括注释)。源文件不得以部分预处理标记或部分
    注释结尾。每个注释由一个空格字符替换。保留新行字符。除新行之外的每个非空空格
    字符序列是保留还是替换为一个空格字符由实现定义。}
  \item{执行预处理指令,展开宏调用,执行\tm{\_Pragma}一元运算符表达式。如果通过
    标记联接(\ref{lang.ppdir.macro.concat})生成与通用字符名语法匹配的字符序列
    则行为未定义。\tm{\#include}预处理指令使命名的头文件或源文件从阶段1到阶段4递
    归地进行处理。然后删除所有预处理指令。}
  \item{字符常量和字符串文本中的每个源字符集成员和转义序列转换为执行字符集的相应
    成员;如果没有相应的成员,则转换为除空(宽)字符以外的实现定义成员。
    \footnote{实现不需要将所有非对应的源字符转换为相同的执行字符。}}
  \item{连接相邻字符串文本。}
  \item{分隔标记的空白字符不再有意义。每个预处理标记都转换为标记。生成的标记在语
    法和语义上被分析并翻译为翻译单元。}
  \item{解析所有外部对象和函数引用。链接库组件以满足当前翻译中未定义的函数和对象
    的外部引用。所有这样的翻译输出都被收集到一个程序映像中,该映像包含其在执行环
    境中执行所需的信息。}
\end{enumerate}

\fwdref{6.4.3,6.4,6.10,5.2.1.1,6.9。}

\sssect{诊断}{env.concept.trans.diag}
\paragraph{}
如果预处理翻译单元或翻译单元违反任何语法规则或约束,即使行为已明确说明为未定义或
实现定义,合规实现也应至少生成一条诊断消息(以实现定义的方式标识)。在其他情况下
不需要生成诊断消息。\footnote{其目的是,一个实现应该识别每一个违规行为的性质,并
在可能的情况下定位每一个违规行为。当然,只要有效程序仍然被正确地翻译,实现就可以
自由地产生任意数量的诊断。实现还可能成功地翻译一个无效程序。}

\paragraph{}
\ex* 实现需要为翻译单元生成一条诊断:
\begin{lstlisting}
    char i;
    int i;
\end{lstlisting}
因为在这些情况下,如果本国际标准中的措辞将结构的行为描述为即是约束错误,又会导致
未定义行为,则应诊断约束错误。

\ssect{执行环境}{env.concept.exec}
\paragraph{}
定义两个执行环境:\textit{自由式}和\textit{宿主式}。在这两种情况下,
\textit{程序启动}都发生在执行环境调用指定C函数时。所有具有静态存储期的对象应在程
序启动前初始化(设置其初始值)。未指定初始化方式和时间。\textit{程序终止}将控制
权返回到执行环境。

\fwdref{6.2.4,6.7.9。}

\sssect{自由环境}{env.concept.exec.free}
\paragraph{}
在一个独立的环境中(在这种环境中C程序可能没有任何操作系统的有助下执行),在程序
启动时调用的函数的名称和类型是实现定义的。除第\ref{conform}章所要求的最小集合外,
任何可供独立程序使用的库都由实现定义。

\paragraph{}
在独立环境中程序终止的效果由实现定义。

\sssect{宿主环境}{env.concept.exec.hosted}
\paragraph{}
不需要提供宿主环境,但如果存在,则应符合以下规范。

\ssssect{程序启动}{env.concept.exec.hosted.start}
\paragraph{}
程序启动时调用的函数名为\texttt{main}。实现没有为此函数声明原型。应该定义其返回
类型为\tm{int},且没有参数:
\begin{lstlisting}
    int main(void) { /* ... */ }
\end{lstlisting}
或者带两个参数(此处称为\texttt{argc}和\texttt{argv},尽管可以使用任意名字,因为
它们是局部于定义它们的函数的):
\begin{lstlisting}
    int main(int argc, char *argv[]) { /* ... */ }
\end{lstlisting}
或其等价形式;\footnote{因此,\tm{int}可以替换成定义为\tm{int}的类型定义名,或者
\texttt{argv}的类型可以写成\texttt{char ** argv},等等。} 或其他实现定义方式。

\paragraph{}
如果声明,则\texttt{main}函数的参数应遵守以下约束:
\begin{itemize}
  \item{\texttt{argc}的值应该非负。}
  \item{\texttt{argv[argc]}应该为零指针。}
  \item{如果\texttt{argc}的值大于零,则数组成员\texttt{argv[0]}到
    \texttt{argv[argc-1]}(含)应包含指向字符串的指针,这些字符串在程序启动之前
    由宿主环境给定由实现定义的值。其目的是向程序提供程序启动之前确定的宿主环境中
    其他位置的信息。如果主机环境不能提供大小写字母的字符串,则实现应确保以小写形
    式接收字符串。}
  \item{如果\texttt{argc}的值大于$0$,\texttt{argv[0]}指向的字符串表示
    \textit{程序名};如果程序名在主机环境中不可用,则\texttt{argv[0][0]}应该为空
    字符。如果\texttt{argc}的值大于$1$,则\texttt{argv[1]}到
    \texttt{argv[argc-1]}指向的字符串表示\textit{程序参数}。}
  \item{\texttt{argc}和\texttt{argv}参数以及\texttt{argv}数组指向的字符串应该可
    以由程序修改,并在程序启动和程序终止之间保留其最后存储值。}
\end{itemize}

\ssssect{程序执行}{env.concept.exec.hosted.exec}
\paragraph{}
在宿主环境中,程序可以使用库章(第\ref{lib}章)中描述的所有函数、宏、类型定义和
对象。

\ssssect{程序终止}{env.concept.exec.hosted.term}
\paragraph{}
如果\texttt{main}函数的返回类型是与\tm{int}兼容的类型,则从对\texttt{main}函数的
初始调用返回等同于用\texttt{main}函数返回值作为参数调用\texttt{exit}函数;
\footnote{根据第\ref{lang.concept.storage}节的规定,在前一种情况下\texttt{main}
中声明的自动存储期对象的生命期将会结束,即使在后一种情况下未结束。}
达到终止\texttt{main}函数的\texttt{\}}时返回值为$0$。如果返回类型与\tm{int}不兼
容,则返回到宿主环境的终止状态未指定。

\fwdref{7.1.1,7.22.4.4}

\sssect{程序执行}{env.concept.exec.exec}
\paragraph{}
本国际标准中的语义描述叙述了与优化问题无关的抽象机行为。

\paragraph{}
访问易失性对象、修改对象、修改文件或调用执行任何这些操作的函数都是更改执行环境状
态的\textit{副作用}。\footnote{二进制浮点运算的IEC 60559标准要求某些用户可访问的
状态标志和控制模式。浮点运算隐式设置状态标志;模式影响浮点运算的结果值。支持这种
浮点状态的实现需要将对它的更改视为副作用 -- 有关详细信息参见附录\ref{fparith}。
浮点环境库\texttt{<fenv.h>}提供了一种编程工具,用于指示这些副作用何时重要,从而
在其他情况下释放实现。} 表达式的求值通常包括值计算和副作用的开始。左值表达式的值
计算包括确定指定对象的标识。

\paragraph{}
\textit{前序}(\textit{sequenced before})指单线程执行的求值间的非对称、传递、配
对的关系,可推导出求值间的偏序。对任意两个求值$A$和$B$,如果$A$前序于$B$,则$A$
的执行应先于$B$的执行。(相反,如果$A$前序于$B$,则$B$\textit{后序}于$A$)。如果
$A$不是前序或后序于$B$,则$A$和$B$无序。当$A$前序或后序于$B$但未指明哪一个时,求
值$A$和$B$为不确定地排序。\footnote{无序计算的执行可以交错进行。不确定顺序的计算
不能交错,但可以以任何顺序执行。} 表达式$A$和$B$之间存在\textit{序列点}意味着与
$A$相关的每一个数值计算和副作用都是前序于与$B$相关的每一个数值计算和副作用。(附
录\ref{seq}给出序列点总结)。

\paragraph{}
在抽象机中,所有表达式都按照语义指定的方式进行计算。如果一个实际实现能够推断出它
的值没有被使用,并且不产生任何需要的副作用(包括调用函数或访问易失性对象所引起的
副作用),那么它就不需要计算表达式的一部分。

\paragraph{}
当抽象机的处理因接收到信号而中断时,既不是无锁原子对象,也不是
\texttt{volatile sig\_atomic\_t}类型的对象的值未指定,如同浮点环境的状态一样。处
理程序所修改的,既不是无锁原子对象,也不是\texttt{volatile sig\_atomic\_t}类型的
任何对象的值,如果由处理程序修改而未还原到其原始状态,在处理程序退出时变为不确定
值,如同浮点环境的状态一样。

\paragraph{}
对合规实现的最低要求是:
\begin{itemize}
  \item{对易失性对象的访问严格按照抽象机的规则进行求值。}
  \item{在程序终止时,写入文件的所有数据应与根据抽象语义执行程序的结果相同。}
  \item{交互设备的输入和输出动态应按照\ref{lib.io.file}的规定进行。这些要求的目
  的是未缓冲或行缓冲输出尽早出现,以确保提示消息在等待输入的程序之前实际出现。}
\end{itemize}
这是程序的\textit{可观察行为}。

\paragraph{}
交互设备的构成由实现定义。

\paragraph{}
抽象语义和实际语义之间的更严格的对应关系可以通过每个实现来定义。

\paragraph{}
\ex 实现可以定义抽象语义和实际语义之间一一对应的关系:在每个序列点上,实际对象的
值将与抽象语义指定的值一致。关键字\tm{volatile}将是多余的。

\paragraph{}
或者,一个实现可以在每个翻译单元内执行各种优化,使得只有在跨越翻译单元边界进行函
数调用时,实际语义才会与抽象语义一致。在这种实现中,当调用函数和被调用函数位于不
同翻译单元中,在进入每个函数和函数返回时,所有外部链接对象的值以及通过其中的指针
可访问的所有对象的值将与抽象语义一致。此外,在进入每个这样的函数时,被调用函数的
参数值以及通过其中的指针可访问的所有对象的值将与抽象语义一致。在这种类型实现中,
由\texttt{signal}函数激活的中断服务例程所引用的对象需要显式地指定易失性存储,以
及其他实现定义的限制。

\paragraph{}
\ex 执行以下片断时
\begin{lstlisting}
    char c1, c2;
    /* ... */
    c1 = c1 + c2;
\end{lstlisting}
``整型提升''要求抽象机将每个变量的值提升为\tm{int}大小,然后将两个\tm{int}相加并
截断。如果可以在不溢出的情况下对两个字符做加法,或者在静默地溢出环绕以产生正确的
结果,则实际执行只需要产生相同的结果,可能会忽略提升。

\paragraph{}
\ex 类似的在以下片断中
\begin{lstlisting}
    float f1, f2;
    double d;
    /* ... */
    f1 = f2 * d;
\end{lstlisting}
如果实现可以确定结果与使用双精度算法执行的结果相同,则可以使用单精度算法执行乘法
(例如,如果将\texttt{d}替换为具有类型\texttt{double}的常量$2.0$)。

\paragraph{}
\ex 使用宽寄存器的实现必须注意遵守适当的语义。值与它们是在寄存器中还是在内存中表
示无关。例如,不允许寄存器的隐式溢出更改值。此外,还需要显式存储和加载来四舍五入
到存储类型的精度。特别是,强制转换和分配需要执行其指定的转换。对于片断
\begin{lstlisting}
    double d1, d2;
    float f;
    d1 = f = expression;
    d2 = (float) expression;
\end{lstlisting}
分配给d1和d2的值需要转换为float。

\paragraph{}
\ex 由于精度和范围的限制,浮点表达式的重新排列经常受到限制。由于舍入误差,即使在
没有溢出和下溢的情况下,该实现通常也不能应用加法或乘法的数学关联规则,也不能应用
分布规则。同样,实现通常不能替换十进制常量来重新排列表达式。在下面的片段中,由实
数数学规则建议的重新排列通常是无效的(见F.9)。
\begin{lstlisting}
    double x, y, z;
    /* ... */
    x = (x * y) * z;   // not equivalent to x *= y * z;
    z = (x - y) + y;   // not equivalent to z = x;
    z = x + x * y;     // not equivalent to z = x * (1.0 + y);
    y = x / 5.0;       // not equivalent to y = x * 0.2;
\end{lstlisting}

\paragraph{}
\ex 为了说明表达式的分组行为,在下面的片段中
\begin{lstlisting}
    int a, b;
    /* ... */
    a = a + 32760 + b + 5;
\end{lstlisting}
由于这些运算符的结合性和优先级,表达式语句的行为与
\begin{lstlisting}
    a = (((a + 72760) + b) + 5);
\end{lstlisting}
相同。因此,求和\texttt{(a + 32760)}的结果接着加上\texttt{b},其结果再加上
\texttt{5},产生的结果赋值给\texttt{a}。在一个溢出产生显式陷阱和\tm{int}的值表示
范围是$[-32768, +32767$的机器上,实现不能将该表达式重写成
\begin{lstlisting}
    a = ((a + b) + 32765);
\end{lstlisting}
因为如果\texttt{a}和\texttt{b}的值对应的为$-32754$和$-15$的话则\texttt{a + b}的
和会产生陷阱,而原始表达式不会;表达式也不能写成
\begin{lstlisting}
    a = ((a + 32765) + b);
\end{lstlisting}
或
\begin{lstlisting}
    a = (a + (b + 32765));
\end{lstlisting}
因为\texttt{a}和\texttt{b}的值可能对应的分别为$4$和$8$或$-17$和$12$。然而在一台
溢出默认产生某个值和正负溢出相互抵消的机器上,以上表达式语句可以由实现重写成以上
任一种方式,因为会产生相同结果。

\paragraph{}
\ex 表达式的分组不能完全确定其求值。在下面的片段中
\begin{lstlisting}
    #include <stdio.h>
    int sum;
    char *p;
    /* ... */
    sum = sum * 10 - '0' + (*p++ = getchar());
\end{lstlisting}
表达式语句如同写成以下进行分组
\begin{lstlisting}
    sum = (((sum * 10) - '0') + ((*(p++)) = (getchar())));
\end{lstlisting}
但\texttt{p}的实际自增可能发生在前一个序列点和后一个序列点(\tm{;}处)之间任意时
间,且\texttt{getchar}的调用可以在需要其返回值前任一点发生。

\fwdref{6.5,6.7.3,6.8,7.6,7.14,7.21.3。}

\sssect{多线程执行与数据竞争}{env.concept.exec.thread}
\paragraph{}
在宿主实现下,程序可以有多个\textit{执行线程}(或简称\textit{线程})同时运行。每
个线程的执行按照本标准其余部分的定义进行。整个程序的执行包括所有线程的执行。
\footnote{执行通常可以看作是所有线程的交错。但是,例如某些种类的原子操作,允许执
行与下面描述的简单交错不一致的操作。} 在独立的实现中,一个程序是否可以有多个执行
线程由实现定义。

\paragraph{}
在特定点上线程$T$可见的对象值为对象的初始值、$T$存储在对象中的值或按以下规则由其
他线程存储在对象中的值。

\paragraph{}
\notes 在某些情况下,可能会有未定义的行为。本节的大部分内容都是为了支持具有明确
和详细可见性约束的原子操作。但是,它还隐式地支持更受限制的程序使用更简单的视图。

\paragraph{}
如果两个表达式计算中的一个修改了内存位置,而另一个读取或修改了相同的内存位置,则
这会发生冲突。

\paragraph{}
标准库定义了许多\textit{原子操作}(\ref{lib.atom})和互斥体上的操作
(\ref{lib.thread.mutex}),这些操作被专门标识为同步操作。这些操作在使一个线程中
的分配对另一个线程可见方面发挥着特殊的作用。一个或多个内存位置上的
\textit{同步操作}可以是\textit{获取操作}、\textit{释放操作}、获取和释放操作,也
可以是\textit{消耗操作}。没有关联内存位置的同步操作是一个\textit{围栏},可以是获
取围栏、释放围栏,也可以是获取围栏和释放围栏。此外,还有不是同步操作的
\textit{宽松的原子操作},和原子\textit{读-修改-写}操作,它们具有特殊性。

\paragraph{}
\notes 例如,获取互斥体的调用将在组成互斥体的位置上执行获取操作。相应地,释放相
同互斥体的调用将在这些相同的位置上执行释放操作。非正式地说,在$A$上执行释放操作
会使其他内存位置上先前的副作用对之后在$A$上执行获取或消耗操作的其他线程可见。我
们不将宽松的原子操作包括为同步操作,尽管与同步操作一样,它们不会造成数据竞争。

\paragraph{}
对特定原子对象$M$的所有修改都以特定的全序发生,称为$M$的\textit{修改序}。如下定
义,如果$A$和$B$是对原子对象$M$的修改,而$A$发生在$B$之前,则在$M$修改序中$A$应
在$B$之前。

\paragraph{}
\notes 这说明修改序必须遵守``发生在...之前''关系。

\paragraph{}
\notes 每个原子对象都有一个单独的顺序。对于所有对象,不需要将它们组合成一个全序。
一般来说这是不可能的,因为不同的线程可能会观察到不同变量的不一致顺序的修改。

\paragraph{}
由原子对象$M$上的释放操作$A$引导的\textit{释放序列}是$M$的修改序中副作用的最大连
续子序列,其中第一个操作是$A$,每个后续操作要么由执行释放的同一线程执行,要么是
原子读-修改-写操作。

\paragraph{}
某些库调用与另一线程执行的其他库调用\textit{同步}。特别地,对对象$M$执行释放操作
的原子操作$A$与对$M$执行获取操作并读取由$A$引导的释放序列中的任何副作用写入的值
的原子操作$B$同步。

\paragraph{}
\notes 除非在指定的情况下,读取后面的值不一定能确保如下所述的可见性。这种需求有
时会干扰有效的实现。

\paragraph{}
\notes 同步操作的规范定义了一个线程何时读取另一个线程写入的值。对于原子变量,定
义是明确的。给定互斥体上的所有操作都以单个全序发生。每个互斥获取``读取''上一个互
斥释放``所写入的值''。

\paragraph{}
求值$A$\textit{依赖}(\textit{carry a dependency})\footnote{此``依赖''关系是
``前序''关系的子集,并且类似的是严格线程内的。} 于求值$B$,如果:
\begin{itemize}
  \item{$A$的值用作$B$的操作数,除非:
    \begin{itemize}
      \item{$B$是\texttt{kill\_dependency}宏调用,}
      \item{$A$是\tm{\&\&}或\tm{||}运算符的左操作数,}
      \item{$A$是\tm{?:}运算符的左操作数,}
      \item{$A$是\tm{,}运算符的左操作数;}
    \end{itemize}
    或}
  \item{$A$写入标量对象或位域$M$,$B$从$M$读取$A$所写的值,且$A$前序于$B$,或}
  \item{对某些求值$X$,$A$依赖于$X$,$X$依赖于$B$。}
\end{itemize}

\paragraph{}
求值$A$\textit{依赖前序}(\textit{dependency-ordered before})\footnote{``依赖前
序关系''类似于``同步''关系,但使用释放/消耗取代释放/获取} 于求值$B$,如果:
\begin{itemize}
  \item{$A$对原子对象$M$执行释放操作,在另一个线程中,$B$对$M$执行消耗操作,并读
    取由$A$引导的释放序列中的任何副作用写入的值,或}
  \item{对某些求值$X$,$A$依赖前序于$X$,$X$依赖于$B$。}
\end{itemize}

\paragraph{}
求值$A$\textit{线程间先发生}于求值$B$,如果$A$与$B$同步,$A$依赖前序
于$B$,或者,对于某些求值$X$:
\begin{itemize}
  \item{$A$与$X$同步,$X$前序于$B$,}
  \item{$A$前序于$X$,$X$线程间先发生于$B$,或}
  \item{$A$线程间先发生于$X$,$X$线程间先发生于$B$。}
\end{itemize}

\paragraph{}
\notes ``线程间先发生于''关系描述了``前序''、``同步''和``依赖前序''关系的任意连
接,但有两个例外。第一个例外是不允许连接以紧跟``前序''的``依赖前序''结尾。此限制
的原因是,参与``依赖前序''关系的消耗操作仅对与此消耗操作实际依赖的操作提供相关的
排序。这种限制仅适用于这种连接的末尾的原因是,任何后续的释放操作都将为先前的消耗
操作提供所需的顺序。第二个例外是不允许连接完全由``前序''组成。这种限制的原因是:
(1)允许``线程间前发生''是传递封闭的;(2)下面定义的``先发生''关系提供了完全由
``前序''组成的关系。

\paragraph{}
求值$A$\textit{先发生}(\textit{happens before})于求值$B$,如果$A$前序于$B$或者
$A$线程间先发生于$B$。

\paragraph{}
对象$M$相对$M$上的值计算$B$的\textit{可见副作用}$A$满足条件:
\begin{itemize}
  \item{$A$先发生于$B$,且}
  \item{不存在$M$上的副作用$X$使得$A$先发生于$X$,$X$先发生于$B$。}
\end{itemize}
由求值$B$所确定的非原子对象$M$的值,应该是由可见副作用$A$所存值。

\paragraph{}
\notes 如果对哪个对非原子对象的副作用可见有歧义,那么就有一个数据竞争,则行为是
未定义的。

\paragraph{}
\notes 这表明对普通变量的操作不会被可见地重新排序。在没有数据竞争的情况下,这实
际上是无法检测到的,但是有必要确保此处定义的数据竞争在使用原子的适当限制下,与简
单的交错(顺序一致)执行中的数据竞争相对应。

\paragraph{}
原子对象$M$上的\textit{副作用可见序列},相对于$M$的值计算$B$,是$M$的修改序中的
最大连续副作用子序列,其中第一个副作用相对于$B$可见,且对于每个后续副作用,$B$发
生在它之前的情况并非如此。由求值$B$确定的原子对象$M$的值,应为$M$的相对于$B$的可
见序列中的某个操作存储的值。此外,如果原子对象$M$的值计算$A$发生于$M$的值计算$B$
之前,并且$A$计算的值对应于由副作用$X$存储的值,则由$B$计算的值应等于由$A$计算的
值,或是由副作用$Y$存储的值,其中$Y$在$M$的修改序中跟在$X$后面。

\paragraph{}
\notes 这实际上禁止编译器将原子操作重新排序到单个对象,即使两个操作都是``宽松''
的加载。通过这样做,我们可以有效地保证由C原子操作可用的大多数硬件提供的``缓存一
致性''。

\paragraph{}
\notes 可见序列取决于``先发生''关系,而这又取决于我们在这里限制的原子载入所观察
到的值。正确的理解是必须存在原子载入与它们观察到的修改的关联,其与适当选择的修改
序以及如上所述的``先发生''关系一起满足这里强加的结果约束。

\paragraph{}
如果一个程序在不同的线程中包含两个相互冲突的动作,那么其执行将包含一个数据竞争,
其中至少一个动作不是原子的,并且两个动作都不会在另一个线程之前发生。任何这样的数
据竞争都会导致未定义行为。

\paragraph{}
\notes 可以证明,正确使用简单互斥和\texttt{memory\_order\_seq\_cst}操作来防止所
有数据竞争,并且不使用其他同步操作的程序,其行为就好像其组成线程执行的操作只是交
错的,对象的每个值计算都是在交错中存储的最后一个值。这通常被称为“顺序一致性”。
然而,这只适用于无数据竞争的程序,并且无数据竞争的程序不能观察大多数不改变单线程
程序语义的程序转换。事实上,大多数单线程程序转换仍然是允许的,因为任何结果表现不
同的程序都必须包含未定义的行为

\paragraph{}
\notes 通常,本标准会排除掉某些编译器变换,这些变换将赋值引入抽象机不会修改的潜
在共享内存位置,因为这样的赋值可能会在抽象机执行未遇到数据竞争的情况下覆盖不同的
线程另一个赋值。这包括覆盖独立内存位置上的相邻成员的数据成员赋值实现。通常也排除
原子载入的重新排序,因为这可能违反``可见序列''规则。

\paragraph{}
\notes 引入对潜在共享内存位置的推测性读取的转换可能不会保持本标准中定义的程序语
义,因为它们可能引入数据竞争。但是,它们在优化编译器的上下文中通常是有效的,该编
译器针对具有定义良好的数据竞争语义的特定机器。对于一个不允许数据竞争或提供硬件数
据竞争检测的假想机器来说,它们是无效的。

\sect{环境考虑}{env.env}
\ssect{字符集}{env.env.charset}
\paragraph{}
应定义两个字符集及其相关的排序序列:用于写源文件的集合(\textit{源字符集})和用
于执行环境中解释的集合(\textit{执行字符集})。每个集合进一步分为一个
\textit{基本字符集},其内容由本节给出,以及一组零个或多个语言环境特定的成员(不
是基本字符集的成员),称为\textit{扩展字符}。组合起来的集合也称为
\textit{扩展字符集}。执行字符集成员的值由实现定义。

\paragraph{}
在字符常量或字符串文字中,执行字符集的成员应由源字符集的相应成员或由反斜杠
\tm{\bs}后跟一个或多个字符组成的\textit{转义序列}表示。所有位都设置为$0$的字节,
称为\textit{零字符},应存在于基本执行字符集中;用于终止字符串。

\paragraph{}
基本源字符集和基本执行字符集都应该具有以下成员:

拉丁字母表的$26$个\textit{大写字母}

\mbox{\qquad\tm{A\ \ B\ \ C\ \ D\ \ E\ \ F\ \ G\ \ H\ \ I\ \ J\ \ K\ \ L\ \ M}}

\mbox{\qquad\tm{N\ \ O\ \ P\ \ Q\ \ R\ \ S\ \ T\ \ U\ \ V\ \ W\ \ X\ \ Y\ \ Z}}

拉丁字母表的$26$个\textit{小写字母}

\mbox{\qquad\tm{a\ \ b\ \ c\ \ d\ \ e\ \ f\ \ g\ \ h\ \ i\ \ j\ \ k\ \ l\ \ m}}

\mbox{\qquad\tm{n\ \ o\ \ p\ \ q\ \ r\ \ s\ \ t\ \ u\ \ v\ \ w\ \ x\ \ y\ \ z}}

$10$个十进制\textit{数字}

\mbox{\qquad\tm{0\ \ 1\ \ 2\ \ 3\ \ 4\ \ 5\ \ 6\ \ 7\ \ 8\ \ 9}}

以下$29$个图形字符

\mbox{\qquad\tm{!\ \ \dq\ \ \#\ \ \%\ \ \&\ \ \sq\ \ (\ \ )\ \ *\ \ +\ \ ,\ \ -\
                \ .\ \ /\ \ :}}

\mbox{\qquad\tm{;\ \ <\ \ =\ \ >\ \ ?\ \ [\ \ \bs\ \ ]\ \ \^\ \ \_\ \ \{\ \ |\
                \ \}\ \ \~}}

空格,表示水平制表符、垂直制表符和换页符的控制字符。源和执行基本字符集的每个成员
的表示应适合于一个字节。在源字符集和执行基本字符集中,上述十进制数字列表中0之后
的每个字符的值应大于前一个字符的值。在源文件中,应该有某种方式指示每行文本结尾;
本国际标准将这种行结尾指示符视为单个新行字符。在基本执行字符集中,应有表示警告、
退格、回车和换行的控制字符。如果在源文件中遇到任何其他字符(标识符、字符常量、字
符串文字、头名称、注释或未转换为标记的预处理标记中的字符除外),则该行为未定义。

\paragraph{}
\textit{字母}指上面定义的大写字母或小写字母;在本国际标准中,该术语不包括其他字
母表中的字母。

\paragraph{}
通用字符名提供命名其他字符的方式。

\fwdref{6.4.3,6.4.4.4,6.10,6.4.5,6.4.9,7.1.1}

\sssect{三字母序列}{env.env.charset.trigraph}
\paragraph{}
在进行任何其他处理之前,以下三个字符序列中的每一个序列(称为三字母序列
\footnote{三字母序列允许输入未在Invariant Code Set中定义的字符,如ISO/IEC 646所
述,后者是$7$位US ASCII代码集的一个子集。})都被替换为对应的单个字符。

\mbox{\qquad\tm{??=\quad\#\qquad\qquad??)\quad]\qquad\qquad??!\quad|}}

\mbox{\qquad\tm{??(\quad[\qquad\qquad??\sq\quad\^\qquad\qquad??>\quad\}}}

\mbox{\qquad\tm{??/\quad\bs\qquad\qquad??<\quad\{\qquad\qquad??-\quad\~}}

不存在其他三字母序列。未开始一个以上列出的三字母序列的每一个\tm{?}不变。

\paragraph{}
\ex
\begin{lstlisting}
    ??=define arraycheck(a, b) a??(b??) ??!??! b??(a??)
\end{lstlisting}
变成
\begin{lstlisting}
    #define arraycheck(a, b) a[b] || b[a]
\end{lstlisting}

\paragraph{}
\ex 以下源行
\begin{lstlisting}
    printf("Eh???/n");
\end{lstlisting}
变成(在替换三字母序列\tm{??/}后)
\begin{lstlisting}
    printf("Eh?\n");
\end{lstlisting}

\sssect{多字节字符}{env.env.charset.multibyte}
\paragraph{}
源字符集可能包含多字节字符,用于表示扩展字符集。执行字符集也可能包括多字符,不需
要与源字符集相同的编码。对这两种字符集,以下应该成立:
\begin{itemize}
  \item{基本字符集应该存在,且每个字符应该编码成单字节。}
  \item{额外成员是否存在,其语义和表示是语言环境特定的。}
  \item{多字节字符集可能具有\textit{状态相关编码}(state-dependent encoding),
    其中每个多字节字符序列以\textit{初始升档状态}(\textit{initial shift state})
    开始,并在遇到该序列中特定的多字节字符时进入其他语言环境特定的\textit{升档状
    态}(\textit{shift state})。当处于升档状态下时,所有单字节字符保留其常规解
    释,不更改升档状态。序列中后续字节的解释是当前升档状态的函数。}
  \item{一个所有位为零的字节应该解释成零字符,无关于升档状态。这样的字节不应该作
    为其他多字节字符的任何部分。}
\end{itemize}

\paragraph{}
对源文件,以下应该成立:
\begin{itemize}
  \item{标识符、注释、字符串、字符常量或头名称应该以升档状态开始并结束。}
  \item{标识符、注释、字符串、字符常量或头名称应该包含有效多字节字符序列。}
\end{itemize}

\ssect{字符显示语义}{env.env.display}
\paragraph{}
\textit{活动位置}(\textit{active position})指显示设备上\texttt{fputc}函数输出
的下一个字符出现的位置。向显示设备写打印字符(由\texttt{isprint}函数定义)的目的
是在活动位置显示该字符的图形表示,并在当前行上将活动位置向前移动。书写方向是语言
环境特定的。如果活动位置位于该行的最后(如果有的话),则显示设备的行为未定义。

\paragraph{}
表示执行字符集中的非图形字符的字母转义序列的目的是在显示设备上产生如下行为:
\begin{itemize}
  \item{\tm{\bs a}(警告)产生视觉或听觉警告,不改变活动位置。}
  \item{\tm{\bs b}(退格)将活动位置移动到当前行的前一个位置。如果活动位置在当前
    行的初始位置,则显示设备行为未定义。}
  \item{\tm{\bs f}(分页)将活动位置移动到下一个逻辑页的初始位置。}
  \item{\tm{\bs n}(新行)将活动位置移动到下一行初始位置。}
  \item{\tm{\bs r}(回车)将活动位置移动到当前行初始位置。}
  \item{\tm{\bs t}(水平制表符)将活动位置移动到当前行下一个水平制表位置。如果活
    动位置位于当前行的最后一个定义的水平制表位,则显示设备行为未定义。}
  \item{\tm{\bs v}(垂直制表符)将活动位置移动到当前行下一个垂直制表位置。如果活
    动位置位于当前行的最后一个定义的垂直制表位,则显示设备行为未定义。}
\end{itemize}

\paragraph{}
每一个这种转义序列应该产生一个唯一实现定义的值,可存储于单个\tm{char}对象中。文
本文件中的外部表示不需要等价于内部表示,且不属于本国际标准定义范围。

\fwdref{7.4.1.8,7.21.7.3}

\ssect{信号与中断}{env.env.sig}
\paragraph{}
函数应该实现成它们可能会在任何时间由信号中断或由信号处理程序调用,或两者均可,而
不改变较早(但仍活跃的)调用控制流以及函数返回值或自动存储期对象。所有这样的对象应
该在每次调用的基础上维持在\textit{函数镜像}(构成函数执行表示的指令)外。

\ssect{环境限制}{env.env.limit}
\paragraph{}
翻译和执行环境都会限制语言翻译和库的实现。以下总结了合规实现语言相关的环境限制;
库相关的限制在第\ref{lib}章讨论。

\sssect{翻译限制}{env.env.limit.trans}
\paragraph{}
实施应能够翻译和执行至少一个程序,该程序至少包含以下每一种限制的一个实例:
\footnote{实现应该避免任何可能的情况下强加固定的翻译限制。}
\begin{itemize}
  \item{块嵌套127层}
  \item{条件包含嵌套63层}
  \item{一个声明中修改算术、结构、联合或\tm{void}类型的12个指针,数组和函数声明
     子(任意组合)}
  \item{全声明子中括号声明子嵌套63层}
  \item{全表达式中括号表达式嵌套63层}
  \item{内部标识符或宏名中63个有效字符(每个通用字符名或扩展源字符当作单个字符}
  \item{外部标识符中31个有效字符(每个指定短标识符0000FFFF或更少的通用字符名当作
    6个字符,每个指定短标识符00010000或更多的通用字符名当作10个字符,如果存在,
    每个扩展源字符中字符个数与对应通用字符名一样)\footnote{见``未来语言方向''
    (\ref{lang.dir.extn})。}}
  \item{一个翻译单元中4095个外部标识符}
  \item{块内声明的块作用域标识符511个}
  \item{一个预处理翻译单元中同时定义宏标识符4095个}
  \item{一个函数定义参数127个}
  \item{一个函数调用参数127个}
  \item{一个宏定义参数127个}
  \item{一个宏调用参数127个}
  \item{逻辑源行中4095个字符}
  \item{字符串文本中4095个字符(连接以后)}
  \item{一个对象中65535个字节(仅限宿主环境)}
  \item{包含文件15层嵌套}
  \item{\tm{switch}语句中1023个\tm{case}标签(不包括嵌套的\tm{switch})}
  \item{单个结构或联合中1023个成员}
  \item{单个枚举中1023个枚举常量}
  \item{单个结构声明列表中63层嵌套结构或联合定义}
\end{itemize}

\sssect{数值限制}{env.env.limit.num}
\paragraph{}
实现需要记录该章节中指定的所有限制,这些限制在头文件\tm{<limits.h>}和
\tm{<float.h>}中指定。其他限制在\tm{<stdint.h>}中指定。

\fwdref{7.20}

\ssssect{整型大小<limits.h>}{env.env.limits.num.int}
\paragraph{}
下面给出的值应替换为适于在\tm{\#if}预处理指令中使用的常量表达式。此外,除了
\tm{CHAR\_BIT}和\tm{MB\_LEN\_MAX}之外,以下应替换为与表达式类型相同的表达式,该
表达式是根据整数提升转换的相应类型的对象。其实现定义值应等于或大于所示数值(绝对
值),符号相同。
\begin{itemize}
  \item{非位字段的最小对象的位数(字节)
    \begin{itemize}
      \itemkv{CHAR\_BIT}{8}
    \end{itemize}}
  \item{\tm{signed char}类型对象的最小值
    \begin{itemize}
      \itemkv[$-(2^7-1)$]{SCHAR\_MIN}{-127}
    \end{itemize}}
  \item{\tm{signed char}类型对象的最大值
    \begin{itemize}
      \itemkv[$2^7-1$]{SCHAR\_MAX}{+127}
    \end{itemize}}
  \item{\tm{unsigned char}类型对象的最大值
    \begin{itemize}
      \itemkv[$2^8-1$]{UCHAR\_MAX}{+127}
    \end{itemize}}
  \item{\tm{char}类型对象的最小值
    \begin{itemize}
      \itemkv[见以下]{CHAR\_MIN}{}
    \end{itemize}}
  \item{\tm{char}类型对象的最大值
    \begin{itemize}
      \itemkv[见以下]{CHAR\_MAX}{}
    \end{itemize}}
  \item{对于任何支持的语言环境,多字节字符中的最大字节数
    \begin{itemize}
      \itemkv{MB\_LEN\_MAX}{1}
    \end{itemize}}
  \item{\tm{short int}类型对象的最小值
    \begin{itemize}
      \itemkv[$-(2^{15}-1)$]{SHRT\_MIN}{-32767}
    \end{itemize}}
  \item{\tm{short int}类型对象的最大值
    \begin{itemize}
      \itemkv[$2^{15}-1$]{SHRT\_MAX}{+32767}
    \end{itemize}}
  \item{\tm{unsigned short int}类型对象的最大值
    \begin{itemize}
      \itemkv[$2^{16}-1$]{USHRT\_MAX}{65535}
    \end{itemize}}
  \item{\tm{int}类型对象的最小值
    \begin{itemize}
      \itemkv[$-(2^{15}-1)$]{INT\_MIN}{-32767}
    \end{itemize}}
  \item{\tm{int}类型对象的最大值
    \begin{itemize}
      \itemkv[$2^{15}-1$]{INT\_MAX}{+32767}
    \end{itemize}}
  \item{\tm{unsigned int}类型对象的最大值
    \begin{itemize}
      \itemkv[$2^{16}-1$]{UINT\_MAX}{65535}
    \end{itemize}}
  \item{\tm{long int}类型对象的最小值
    \begin{itemize}
      \itemkv[$-(2^{31}-1)$]{LONG\_MIN}{-2147483647}
    \end{itemize}}
  \item{\tm{long int}类型对象的最大值
    \begin{itemize}
      \itemkv[$2^{31}-1$]{LONG\_MAX}{+2147483647}
    \end{itemize}}
  \item{\tm{unsigned long int}类型对象的最大值
    \begin{itemize}
      \itemkv[$2^{32}-1$]{ULONG\_MAX}{4294967295}
    \end{itemize}}
  \item{\tm{long long int}类型对象的最小值
    \begin{itemize}
      \itemkv[$-(2^{63}-1)$]{LLONG\_MIN}{-9223372036854775807}
    \end{itemize}}
  \item{\tm{long long int}类型对象的最大值
    \begin{itemize}
      \itemkv[$2^{63}-1$]{LLONG\_MAX}{+9223372036854775807}
    \end{itemize}}
  \item{\tm{unsigned long long int}类型对象的最大值
    \begin{itemize}
      \itemkv[$2^{64}-1$]{ULLONG\_MAX}{18446744073709551615}
    \end{itemize}}
\end{itemize}

\paragraph{}
如果表达式中使用的\tm{char}类型对象值被视为有符号整型,则\tm{CHAR\_MIN}的值应与
\tm{SCHAR\_MIN}的值相同,\tm{CHAR\_MAX}的值应与\tm{SCHAR\_MAX}的值相同。否则,
\tm{CHAR\_MIN}的值应为0,\tm{CHAR\_MAX}的值应与\tm{UCHAR\_MAX}的值相同。
\footnote{见\ref{lang.concept.type}} 值\tm{UCHAR\_MAX}应等于$2^{CHAR\_BIT}-1$。

\fwdref{6.2.6,6.10.1}

\ssssect{浮点类型特征<float.h>}{env.env.limits.num.float}
\paragraph{}
浮点类型的特征是根据一个模型来定义的,该模型描述了浮点数和值的表示,这些值提供了
有关实现的浮点数算法的信息。\footnote{浮点模型旨在澄清每个浮点特征的描述,并且不
要求实现的浮点算法相同。} 以下参数用于定义每个浮点类型的模型:
\begin{itemize}
  \item[]{$s$   \qquad 符号($\pm$)}
  \item[]{$b$   \qquad 指数表示的基数(大于1的整数)}
  \item[]{$e$   \qquad 指数(介于最小值$e_{min}$和最大值$e_{max}$之间的整数)}
  \item[]{$p$   \qquad 精度(以\textit{b}为底的有效位位数)}
  \item[]{$f_k$ \qquad 小于$b$的非负整数(有效位数字)}
\end{itemize}

\paragraph{}
\textit{浮点数(x)}使用以下模型定义:
\begin{equation*}
  x = sb^e\sum_{k=1}^p f_k b^{-k}, e_{min} \le e \le e_{max}
\end{equation*}

\paragraph{}
除了标准化的浮点数(如果$x\neq0$,则$f_1>0$),浮点类型还可以包含其他类型的浮点
数,例如\textit{次标准化的浮点数}($x\neq0,e=e_{min},f_1=0$)和\textit{非标准
化的浮点数}($x\neq0,e>e_{min},f_1=0$),以及非浮点数的值,例如无穷大和NaN。
\textit{NaN}是表示非数字(Not-a-Number)。一个\textit{安静的NaN}(\textit{quiet
NaN})几乎在每个算术运算中传播,而不引发浮点异常;在作为算术操作数时信号NaN通常
引发浮点异常。\footnote{IEC 60559:1989规定了安静的和信号NaN。对于不支持IEC
60559:1989的实现,术语安静的NaN和信号NaN旨在适用于具有类似行为的编码。}

\paragraph{}
一个实现可以给零和不是浮点数的值(如无穷大和NaN)一个符号,或者可以不带符号。凡
该类数值为无符号,本国际标准中任何要求查询该符号的地方应产生未指明的标志,而设置
该标志的任何要求应被忽略。

\paragraph{}
浮点类型可表示值的最小范围是该类型中可表示的最小负有限浮点数到该类型中可表示的最
大正有限浮点数。此外,如果一个类型中可以表示负无穷,则该类型的范围将扩展到所有负
实数;同样,如果一个类型中可以表示正无穷,则该类型的范围将扩展到所有正实数。

\paragraph{}
返回浮点结果的\tm{<math.h>}和\tm{<complex.h>}中的浮点运算(\tm{+、-、*、/})和库
函数的精度是实现定义的,正如\tm{<stdio.h>}、\tm{<stdlib.h>}和\tm{<wchar.h>}中库
函数执行的浮点内部表示和字符串表示之间转换的精度一样。实现可能会声明精度未知。

\paragraph{}
除\tm{FLT\_ROUNDS}外,\tm{<float.h>}头中的所有整数值都应是适用于\tm{\#if}预处理
指令的常量表达式;所有浮点值都应是常量表达式。除\tm{DECIMAL\_DIG}、
\tm{FLT\_EVAL\_METHOD}、\tm{FLT\_RADIX}和\tm{FLT\_ROUNDS}外,所有三种浮点类型都
有单独的名称。除\tm{FLT\_EVAL\_METHOD}和\tm{FLT\_ROUNDS}外,所有值都提供了浮点
模型表示。

\paragraph{}
浮点加法的舍入模式由实现定义的\tm{FLT\_ROUNDS}值刻画:\footnote{通过
\tm{<fenv.h>}中的函数\tm{fesetround},\tm{FLT\_ROUNDS}的求值正确反映了舍入模式的
任何执行时间变化。}
\begin{itemize}
  \item[]{\tm{-1}  \qquad 不确定}
  \item[]{\tm{\ 0} \qquad 向零舍入}
  \item[]{\tm{\ 1} \qquad 向最近值舍入}
  \item[]{\tm{\ 2} \qquad 向正无穷舍入}
  \item[]{\tm{\ 3} \qquad 向负无穷舍入}
\end{itemize}
\tm{FLT\_ROUNDS}的所有其他值刻画实现定义的舍入行为。

\paragraph{}
除了赋值和强制转换(删除了所有额外的范围和精度),由具有浮点操作数的运算符生成的
值以及受常规算术转换和浮点常量影响的值将被计算为一种格式,其范围和精度可能大于类
型所需的值。求值格式的使用由实现定义的\tm{FLT\_EVAL\_METHOD}值进行刻画:
\footnote{该计算方法确定涉及所有浮点类型的表达式的计算格式,而不仅仅是实类型。
例如,如果\tm{FLT\_EVAL\_METHOD}为1,则两个浮点\tm{\_Complex}操作数的乘积用
\tm{double \_Complex}格式表示,其实虚部系数以\tm{double}计算。}
\begin{itemize}
  \item[]{\tm{-1}  \qquad 不确定;}
  \item[]{\tm{\ 0} \qquad 所有操作和常量求值仅到其类型的范围和精度;}
  \item[]{\tm{\ 1} \qquad \tm{float}和\tm{double}类型操作和常量求值到\tm{double}
    类型的范围和精度,\\
          \hphantom{\tm{\ 1}} \qquad \tm{long double}类型操作和常量求值到
    \tm{loung double}类型的范围和精度;}
  \item[]{\tm{\ 2} \qquad 所有操作和常量求值到\tm{long double}类型范围和精度。}
\end{itemize}
\tm{FLT\_EVAL\_METHOD}的所有其他负值刻画实现定义行为。

\paragraph{}
次标准数的存在或不存在由\tm{FLT\_HAS\_SUBNORM}、\tm{DBL\_HAS\_SUBNORM}和
\tm{LDBL\_HAS\_SUBNORM}的实现定义值刻画:
\begin{itemize}
  \item[]{\tm{-1}  \qquad 不确定\footnote{如果浮点操作不能一致地将次标准表示解释
    为零或非零,则刻画为不可确定。}}
  \item[]{\tm{\ 0} \qquad 不存在\footnote{如果浮点操作不能从非次标准输入产生次标
    准结果,即使类型格式包括次标准数表示,也刻画为不存在。}(类型不支持次标准数
    )}
  \item[]{\tm{\ 1} \qquad 存在(类型支持次标准数)}
\end{itemize}

\paragraph{}
以下列表中给出的值应替换为具有使用实现定义值的常量表达式,这些值的大小(绝对值)
应大于或等于所示值,符号相同:
\begin{itemize}
  \item{指数表示的基数$b$
    \begin{itemize}
      \itemkv{FLT\_RADIX}{2}
    \end{itemize}}
  \item{浮点有效位中,以\tm{FLT\_RADIX}为基数的位数$p$
    \begin{itemize}
      \itemkv{FLT\_MANT\_DIG}{}
      \itemkv{DBL\_MANT\_DIG}{}
      \itemkv{LDBL\_MANT\_DIG}{}
    \end{itemize}}
  \item{十进制位数$n$使得基数为$p$,$b$位数的任何浮点数都可以四舍五入为具有$n$位
    十进制数字的浮点数而再转回后不改变其值,
    \[
      \begin{cases}
        \begin{tabular}{ll}
          $p\log_{10}b$ & $b$为$10$的幂次                                     \\
          $\lceil 1 + p\log_{10}b \rceil$ & 否则
        \end{tabular}
      \end{cases}
    \]
    \begin{itemize}
      \itemkv{FLT\_DECIMAL\_DIG}{6}
      \itemkv{DBL\_DECIMAL\_DIG}{10}
      \itemkv{LDBL\_DECIMAL\_DIG}{10}
    \end{itemize}}
  \item{十进制位数$n$使支持的最宽浮点类型中的基数为$p_{max}$,$b$位数表示的任何
    浮点数都可以四舍五入为具有$n$个十进制数位数的浮点数而再转回后不改变其值,
    \[
      \begin{cases}
        \begin{tabular}{ll}
          $p_{max}\log_{10}b$ & $b$为$10$的幂次                               \\
          $\lceil 1 + p_{max}\log_{10}b \rceil$ & 否则
        \end{tabular}
      \end{cases}
    \]
    \begin{itemize}
      \itemkv{DECIMAL\_DIG}{10}
    \end{itemize}}
  \item{十进制位数$q$使得任何$q$位十进制浮点数都可以四舍五入为具有$b$位数以
    $p$为基数的浮点数而再转回后不改变其值,
    \[
      \begin{cases}
        \begin{tabular}{ll}
          $p\log_{10}b$ & $b$为$10$的幂次                                     \\
          $\lfloor (p - 1)\log_{10}b \rfloor$ & 否则
        \end{tabular}
      \end{cases}
    \]
    \begin{itemize}
      \itemkv{FLT\_DIG}{6}
      \itemkv{DBL\_DIG}{10}
      \itemkv{LDBL\_DIG}{10}
    \end{itemize}}
  \item{最小负整数使得\tm{FLT\_RADIX}取该整数次幂减一后为标准浮点数$e_{min}$
    \begin{itemize}
      \itemkv{FLT\_MIN\_EXP}{}
      \itemkv{DBL\_MIN\_EXP}{}
      \itemkv{LDBL\_MIN\_EXP}{}
    \end{itemize}}
  \item{最小负整数使得$10$取该整数次幂后在标准浮点数
    $\lceil \log_{10}b^{e_{min}-1}\rceil$范围中,
    \begin{itemize}
      \itemkv{FLT\_MIN\_10\_EXP}{-37}
      \itemkv{DBL\_MIN\_10\_EXP}{-37}
      \itemkv{LDBL\_MIN\_10\_EXP}{-37}
    \end{itemize}}
  \item{最大整数使得\tm{FLT\_RADIX}取该整数次幂减一后为可表示的有限浮点数
    $e_{max}$
    \begin{itemize}
      \itemkv{FLT\_MAX\_EXP}{}
      \itemkv{DBL\_MAX\_EXP}{}
      \itemkv{LDBL\_MAX\_EXP}{}
    \end{itemize}}
  \item{最大整数使得$10$取该整数次幂后在可表示有限浮点数
      $\lfloor \log_{10}((1 - b^{-p})b^{e_{max}})\rfloor$范围中,
    \begin{itemize}
      \itemkv{FLT\_MAX\_10\_EXP} {+37}
      \itemkv{DBL\_MAX\_10\_EXP} {+37}
      \itemkv{LDBL\_MAX\_10\_EXP}{+37}
    \end{itemize}}
\end{itemize}

\paragraph{}
以下列表中给出的值应替换为大于或等于所示值的实现定义值的常量表达式:
\begin{itemize}
  \item{最大可表示有限浮点数$(1-b^{-p})b^{e_{min}}$
    \begin{itemize}
      \itemkv{FLT\_MAX} {1E+37}
      \itemkv{DBL\_MAX} {1E+37}
      \itemkv{LDBL\_MAX}{1E+37}
    \end{itemize}}
\end{itemize}

\paragraph{}
下表中给出的值应替换为小于或等于所示的实现定义(正)值的常量表达式:
\begin{itemize}
  \item{$1$和给定浮点类型可表示的大于$1$的最小值的差$b^{1-p}$
    \begin{itemize}
      \itemkv{FLT\_EPSILON} {1E-5}
      \itemkv{DBL\_EPSILON} {1E-9}
      \itemkv{LDBL\_EPSILON}{1E-9}
    \end{itemize}}
  \item{最小标准正浮点数$b^{e_{min}-1}$
    \begin{itemize}
      \itemkv{FLT\_MIN} {1E-37}
      \itemkv{DBL\_MIN} {1E-37}
      \itemkv{LDBL\_MIN}{1E-37}
    \end{itemize}}
  \item{最小正浮点数\footnote{如果次标准数的存在或不存在是不确定的,则该值应为不
    大于该类型的最小标准正数的正数。}
    \begin{itemize}
      \itemkv{FLT\_TRUE\_MIN}  {1E-37}
      \itemkv{DBL\_TRUE\_MIN}  {1E-37}
      \itemkv{LDBL\_TRUE\_MIN} {1E-37}
    \end{itemize}}
\end{itemize}

\recprac

\paragraph{}
从\tm{double}(至少)转换到具有\tm{DECIMAL\_DIG}位的十进制并转回应该是恒等函数。

\paragraph{}
\ex 以下描述了符合本国际标准最低要求的假想的浮点表示,以及\tm{<float.h>}头中
\tm{float}类型的适当值:
\begin{equation*}
  x = s16^e\sum_{k=1}^6 f_k 16^{-k}, -31 \le e \le +32
\end{equation*}
\begin{itemize}
  \itemkv{FLT\_RADIX}        {16}
  \itemkv{FLT\_MANT\_DIG}    {6}
  \itemkv{FLT\_ESSILON}      {9.53674316E-07F}
  \itemkv{FLT\_DECIMAL\_DIG} {9}
  \itemkv{FLT\_DIG}          {6}
  \itemkv{FLT\_MIN\_EXP}     {-31}
  \itemkv{FLT\_MIN}          {2.93873588E-39F}
  \itemkv{FLT\_MIN\_10\_EXP} {-38}
  \itemkv{FLT\_MAX\_EXP}     {+32}
  \itemkv{FLT\_MAX}          {3.40282347E+38F}
  \itemkv{FLT\_MAX\_10\_EXP} {+38}
\end{itemize}

\paragraph{}
\ex 下面描述的浮点表示法也满足IEC 60559\footnote{该标准中的浮点模型将$b$的幂从零
相加,因此指数极限的值为此处所示的值减$1$。}中单精度和双精度数的要求,以及
\tm{<float.h>}头中\tm{float}和\tm{double}类型的的适当值:
\begin{equation*}
  x_f = s2^e\sum_{k=1}^{24} f_k 2^{-k}, -125 \le e \le +128
\end{equation*}
\begin{equation*}
  x_d = s2^e\sum_{k=1}^{53} f_k 2^{-k}, -1021 \le e \le +1024
\end{equation*}
\begin{itemize}
  \itemkv               {FLT\_RADIX}         {2}
  \itemkv               {DECIMAL\_DIG}       {17}
  \itemkv               {FLT\_MANT\_DIG}     {24}
  \itemkv[十进制常量]   {FLT\_EPSILON}       {1.19209290E-07F}
  \itemkv[十六进制常量] {FLT\_EPSILON}       {0X1P-23F}
  \itemkv               {FLT\_DECIMAL\_DIG}  {9}
  \itemkv               {FLT\_DIG}           {6}
  \itemkv               {FLT\_MIN\_EXP}      {-125}
  \itemkv[十进制常量]   {FLT\_MIN}           {1.17549435E-38F}
  \itemkv[十六进制常量] {FLT\_MIN}           {0X1P-126F}
  \itemkv[十进制常量]   {FLT\_TRUE\_MIN}     {1.40129846E-45F}
  \itemkv[十六进制常量] {FLT\_TRUE\_MIN}     {0X1P-149F}
  \itemkv               {FLT\_HAS\_SUBNORM}  {1}
  \itemkv               {FLT\_MIN\_10\_EXP}  {-37}
  \itemkv               {FLT\_MAX\_EXP}      {+128}
  \itemkv[十进制常量]   {FLT\_MAX}           {3.40282347E+38F}
  \itemkv[十六进制常量] {FLT\_MAX}           {0X1.fffffeP127F}
  \itemkv               {FLT\_MAX\_10\_EXP}  {+38}
  \itemkv               {DBL\_MANT\_DIG}     {53}
  \itemkv[十进制常量]   {DBL\_EPSILON}       {2.2204460492503131E-16}
  \itemkv[十六进制常量] {DBL\_EPSILON}       {0X1P-52}
  \itemkv               {DBL\_DECIMAL\_DIG}  {17}
  \itemkv               {DBL\_DIG}           {15}
  \itemkv               {DBL\_MIN\_EXP}      {-1021}
  \itemkv[十进制常量]   {DBL\_MIN}           {2.2250738585072014E-308}
  \itemkv[十六进制常量] {DBL\_MIN}           {0X1P-1022}
  \itemkv[十进制常量]   {DBL\_TRUE\_MIN}     {4.9406564584124654E-324}
  \itemkv[十六进制常量] {DBL\_TRUE\_MIN}     {0X1P-1074}
  \itemkv               {DBL\_HAS\_SUBNORM}  {1}
  \itemkv               {DBL\_MIN\_10\_EXP}  {-307}
  \itemkv               {DBL\_MAX\_EXP}      {+1024}
  \itemkv[十进制常量]   {DBL\_MAX}           {1.7976931348623157E+308}
  \itemkv[十六进制常量] {DBL\_MAX}           {0X1.fffffffffffffP1023}
  \itemkv               {DBL\_MAX\_10\_EXP}  {+308}
\end{itemize}
如果支持大于\tm{double}的类型,则\tm{DECIMAL\_DIG}将大于$17$。例如,如果最宽的类
型是使用最小宽度的IEC 60559双扩展格式(64位精度),那么\tm{DECIMAL\_DIG}将是
$21$。

\fwdref{6.10.1,7.3,7.29,7.6,7.22,7.21,7.12}
