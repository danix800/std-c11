\chptr{(信息)序列点}{seq}

\paragraph{}
以下为\ref{env.concept.exec.exec}中所述的序列点:
\begin{itemize}
  \item{函数调用中函数指示符和实参的计算与实际调用之间。
    (\ref{lang.expr.post.call})。}
  \item{以下运算符的第一和第二操作数的计算之间:逻辑与\tm{\&\&}
    (\ref{lang.expr.land});逻辑或\tm{||}(\ref{lang.expr.lor});逗号\tm{,}
    (\ref{lang.expr.comma})。}
  \item{条件运算符\tm{?:}的第一操作数和第二或第三操作数中任一个的计算之间
    (\ref{lang.expr.cond})。}
  \item{全声明子之后:声明子(\ref{lang.dcl.decl})。}
  \item{在全表达式的计算和下一个要计算的全表达式之间。以下为全表达式:不属于复合
    字面值(\ref{lang.dcl.init})的初始化;表达式语句中的表达式
    (\ref{lang.stmt.expr});选择语句的控制表达式(\tm{if}或\tm{switch})
    (\ref{lang.stmt.sel});\tm{while}或\tm{do}语句的控制表达式
    (\ref{lang.stmt.iter});\tm{for}语句的每个(可选)表达式
    (\ref{lang.stmt.iter.for});\tm{return}语句的(可选)表达式
    (\ref{lang.stmt.jmp})。}
  \item{库函数返回之前(\ref{lib.intro.use})。}
  \item{关联于每个格式化的输入/输出函数转换说明符的操作之后(\ref{lib.io.fmt},
    \ref{lib.wchar.fmt})。}
  \item{每次调用比较函数之前和之后,以及对比较函数的任何调用和作为参数传递给该调
    用的对象的任何移动之间(\ref{lib.util.search})。}
\end{itemize}
