\chapter*{前言}
\addcontentsline{toc}{chapter}{前言}
\markboth{Introduction}{}

\paragraph{}
ISO(国际标准化组织)和IEC(国际电工委员会)构成了全球标准化的专门体系。作为ISO
或IEC成员的国家机构通过各自组织成立的技术委员会参与国际标准的制定,以处理特定技
术活动领域。国际标准化组织(ISO)和国际电工委员会(IEC)技术委员会在共同利益领域
开展合作。与国际标准化组织和国际电工委员会联络的其他政府和非政府国际组织也参与了
这项工作。在信息技术领域,ISO和IEC建立了一个联合技术委员会,即ISO/IEC JTC 1。

\paragraph{}
国际标准按照ISO/IEC指令第2部分给出的规则起草。

\paragraph{}
联合技术委员会的主要任务是制定国际标准。联合技术委员会通过的国际标准草案将分发给
各国机构进行表决。作为一项国际标准的出版需要至少$75\%$的国家机构投票通过。

\paragraph{}
请注意,本文件的某些要素可能是专利权的主体。国际标准化组织和国际电工委员会不负责
识别任何或所有此类专利权。

\paragraph{}
ISO/IEC 9899由联合技术委员会ISO/IEC JTC 1,\textit{Information technology},小组
委员会SC 22,\textit{Programming languages,their environments and system
software interfaces}。

\paragraph{}
第三版取消并替换了第二版ISO/IEC 9899:1999,该版本已经过技术修订。它还包括技术勘
误ISO/IEC 9899:1999/Cor 1:2001、ISO/IEC 9899:1999/Cor 2:2004和
ISO/IEC 9899:1999/Cor 3:2007。相对上一版本的主要变化包括:
\begin{itemize}
  \item{条件(可选)支持特性(包括某些先前的强制特性)}
  \item{}<++>
\end{itemize}

\paragraph{}
第二版主要变更包括:
