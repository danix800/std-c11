\chapter*{前言}
\addcontentsline{toc}{chapter}{前言}
\markboth{Introduction}{}

\paragraph{}
ISO(国际标准化组织)和IEC(国际电工委员会)构成了全球标准化的专门体系。作为ISO
或IEC成员的国家机构通过各自组织成立的技术委员会参与国际标准的制定,以处理特定技
术活动领域。国际标准化组织(ISO)和国际电工委员会(IEC)技术委员会在共同利益领域
开展合作。与国际标准化组织和国际电工委员会联络的其他政府和非政府国际组织也参与了
这项工作。在信息技术领域,ISO和IEC建立了一个联合技术委员会,即ISO/IEC JTC 1。

\paragraph{}
国际标准按照ISO/IEC指令第2部分给出的规则起草。

\paragraph{}
联合技术委员会的主要任务是制定国际标准。联合技术委员会通过的国际标准草案将分发给
各国机构进行表决。作为一项国际标准的出版需要至少$75\%$的国家机构投票通过。

\paragraph{}
请注意,本文件的某些要素可能是专利权的主体。国际标准化组织和国际电工委员会不负责
识别任何或所有此类专利权。

\paragraph{}
ISO/IEC 9899由联合技术委员会ISO/IEC JTC 1,\textit{Information technology},小组
委员会SC 22,\textit{Programming languages,their environments and system
software interfaces}。

\paragraph{}
第三版取消并替换了第二版ISO/IEC 9899:1999,该版本已经过技术修订。它还包括技术勘
误ISO/IEC 9899:1999/Cor 1:2001、ISO/IEC 9899:1999/Cor 2:2004和
ISO/IEC 9899:1999/Cor 3:2007。相对上一版本的主要变化包括:
\begin{itemize}
  \item{条件(可选)支持特性(包括某些先前的强制特性)}
  \item{支持多个执行线程,包括改进的内存序列模型、原子对象和线程局部存储(
    \tm{<stdamico.h>}和\tm{<threads.h>})}
  \item{附加浮点特征宏(\tm{<float.h>})}
  \item{查询和指定对象的对齐方式(\tm{<stdlaign.h>},\tm{<stdlib.h>})}
  \item{Unicode字符和字符串(\tm{<uchar.h>})(最初在ISO/IEC TR 19769:2004中指
    定)}
  \item{泛型表达式}
  \item{静态断言}
  \item{匿名结构和联合}
  \item{无返回函数}
  \item{复数创建宏(\tm{<complex.h>)}
  \item{支持以独占方式打开文件}
  \item{删除了\tm{gets}函数(\tm{<stdio.h>})}
  \item{添加了\tm{aligned\_alloc}、\tm{at\_quick\_exit}和\tm{quick\_exit}函数
    (\tm{<stdlib.h>})}
  \item{(有条件)边界检查接口支持(最初在ISO/IEC TR 24731-1:2007中指定)}
  \item{(有条件)支持条件分析}
\end{itemize}

\paragraph{}
第二版主要变更包括:
\begin{itemize}
  \item{通过双字母词和\tm{<iso646.h>}支持受限字符集(最初在
    ISO/IEC 9899-1990/Amd.1:1995中指定)}
  \item{\tm{<wchar.h>}和\tm{<wctype.h>}中的宽字符库支持(最初在
    ISO/IEC 9899-1990/Amd.1:1995中指定)}
  \item{通过有效类型实现更精确的别名规则}
  \item{受限指针}
  \item{变长数组}
  \item{灵活数组}
  \item{参数数组声明子中的\tm{static}和类型限定}
  \item{\tm{<complex.h>}中的复(虚)数支持}
  \item{\tm{<tgmath.h>}中的泛型数学宏}
  \item{\tm{long long int}类型和库函数}
  \item{扩展整型}
  \item{增加的最小翻译限制}
  \item{\tm{<float.h>}中的额外浮点类型特性}
  \item{移除隐式\tm{int}}
  \item{可靠的整数除法}
  \item{通用字符名(\tm{\bs{}u}和\tm{\bs{}U})}
  \item{扩展标识符}
  \item{十六进制浮点常量和\tm{\%a}与\tm{\%A printf/scanf}转换说明符}
  \item{复合字面值}
  \item{指定初始化}
  \item{\tm{//}注释}
  \item{\tm{<inttypes.h>}和\tm{<stdint.h>}中指定宽度整型和相应的库函数}
  \item{移除隐式函数声明}
  \item{预处理算术在\tm{intmax/uintmax}中完成}
  \item{混合声明与语句}
  \item{选择和迭代语句的新作用域}
  \item{整型常量类型规则}
  \item{整型提升规则}
  \item{可变参数宏}
  \item{\tm{<stdio.h>}和\tm{<wchar.h>}中的\tm{vscanf}函数簇}
  \item{\tm{<math.h>}中的额外数学库函数}
  \item{数学库函数对错误条件的处理(\tm{math\_errhandling})}
  \item{\tm{<fenv.h>}中的浮点环境访问}
  \item{IEC 60559(也称IEC 559或IEEE算术)支持}
  \item{\tm{enum}声明中允许末尾逗号}
  \item{\tm{printf}中允许\tm{\%lf}转换说明符}
  \item{内联函数}
  \item{\tm{<stdio.h>}中的\tm{snprintf}函数簇}
  \item{\tm{<stdbool.h>}中的布尔类型}
  \item{空的宏参数}
  \item{新的结构类型兼容规则(标签兼容性)}
  \item{额外的预定义宏名}
  \item{\tm{\_Pragma}预处理运算符}
  \item{标准pragma}
  \item{\tm{\_\_func\_\_}预定义标识符}
  \item{\tm{va\_copy}宏}
  \item{额外的\tm{strftime}转换说明符}
  \item{LIA兼容性附录}
  \item{在二进制文件的开头取消\tm{ungetc}}
  \item{移除取消的别名数组参数}
  \item{数组到指针的转换不再限于左值}
  \item{对聚合和联合的初始化约束放宽}
  \item{对可移植头名称限制放宽}
  \item{在返回值的函数中不允许不带表达式返回(反之亦然)}
\end{itemize}
