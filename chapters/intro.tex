\chapter*{引言}
\addcontentsline{toc}{chapter}{引言}
\markboth{Introduction}{}

\setcounter{paragraph}{0}

\paragraph{}
随着新设备和扩展字符集的引入,本国际标准可能会增加新功能。语言和库子句中的子类警
告实现者和程序员使用的用法,尽管它们本身是有效的,但可能与将来的添加冲突。

\paragraph{}
某些特征是\textit{过时}的,这意味着在未来修订本国际标准时,可以考虑将其撤回。它
们之所以保留下来是因为它们的广泛使用,但不鼓励在新的实现(用于实现特性)或新程序
(用于语言[6.11]或库特性[7.31])中使用它们。

\paragraph{}
本国际标准分成四大部分:
\begin{itemize}
  \item{基本元素(第1-4章);}
  \item{翻译和执行C程序的环境特性(第5章);}
  \item{语言语法,约束和语义(第6章);}
  \item{标准库(第7章)。}
\end{itemize}

\paragraph{}
文档中举例以说明所述结构的可能形式。提供脚注以强调该章节或本国际标准其他地方所述
规则的后果。引用用于引用其他相关章节。本文档向实施者提供建议或指导。附件提供了附
加信息,并总结了本国际标准中包含的信息。参考书目列出了标准编制过程中参考的文件。

\paragraph{}
语言章(第6章)派生自``The C Reference Manual''。

\paragraph{}
库章(第7章)基于\textit{1984 /usr/group Standard}。

\paragraph{}
负责本标准的工作组(WG14)在万维网的http://www.open-std.org/JTC1/SC22/WG14/上维
护了一个网站,其中包含与本标准相关的其他信息,如在其编制过程中作出的许多决定的理
由,以及缺陷报告和响应的日志。
