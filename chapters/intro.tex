\chapter*{引言}
\addcontentsline{toc}{chapter}{引言}
\markboth{Introduction}{}

\setcounter{paragraph}{0}

\paragraph{}
随着新设备和扩展字符集的引入,本国际标准可能会增加新功能。语言和库中的子章节警告
实现者和程序员某些用法尽管它们本身有效但可能与将来的添加产生冲突。

\paragraph{}
某些特征是\textit{过时}的,这意味着在未来修订本国际标准时,可以考虑将其撤回。它
们之所以保留下来是因为它们的广泛使用,但不鼓励在新的实现(用于实现特性)或新程序
(用于语言[6.11]或库特性[7.31])中使用它们。

\paragraph{}
本国际标准分成四大部分:
\begin{itemize}
  \item{基本要素(第1-4章);}
  \item{翻译和执行C程序的环境的特性(第5章);}
  \item{语言的语法,约束和语义(第6章);}
  \item{标准库(第7章)。}
\end{itemize}

\paragraph{}
提供示例以说明所述结构的可能形式。提供脚注以强调该章节或本国际标准其他地方所述规
则的结论。参考引用指向其他相关章节。推荐实践向实现者提供建议或指导。附录提供了附
加信息,并总结了本国际标准中包含的信息。参考书目列出了标准编制过程中参考的文件。

\paragraph{}
语言章节(第6章)派生自``The C Reference Manual''。

\paragraph{}
库章节(第7章)基于\textit{1984 /usr/group Standard}。

\paragraph{}
负责本标准的工作组(WG14)在万维网的http://www.open-std.org/JTC1/SC22/WG14/上维
护了一个网站,其中包含与本标准相关的其他信息,如在其编制过程中作出的许多决定的理
由,以及缺陷报告和响应的日志。
